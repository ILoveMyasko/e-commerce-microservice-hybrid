\chapter*{Введение} % * не проставляет номер
\addcontentsline{toc}{chapter}{Введение} % вносим в содержание
В условиях стремительного роста рынка электронной коммерции (E-commerce) требования к производительности и отказоустойчивости веб-приложений становятся критическими факторами успеха бизнеса. Современные торговые платформы характеризуются неравномерным профилем нагрузки: периоды штатной работы сменяются резкими пиками активности пользователей (например, во время маркетинговых акций «Черная пятница» или сезонных распродаж), когда трафик может возрастать в десятки раз.

Специфика E-commerce систем заключается в существенном преобладании операций чтения над операциями записи (Read-heavy workload). Пользователи просматривают каталоги, используют поиск и изучают карточки товаров значительно чаще, чем оформляют заказы. В микросервисной архитектуре это порождает проблему «усиления трафика» (Traffic Amplification), когда один запрос клиента к странице товара инициирует цепочку вызовов к множеству внутренних сервисов (каталог, склад, отзывы, рекомендации).

Традиционный подход к построению таких систем на базе архитектурного стиля REST и формата передачи данных JSON имеет ряд ограничений. Текстовая природа JSON приводит к избыточности передаваемых данных, а отсутствие строгой типизации и мультиплексирования в протоколе HTTP/1.1 увеличивает сетевые задержки (Latency), особенно для мобильных клиентов с нестабильным интернет-соединением. В связи с этим актуальной задачей является исследование и разработка гибридных архитектурных решений, сочетающих гибкость агрегации данных (GraphQL) с эффективностью бинарных протоколов межсервисного взаимодействия (gRPC).
Объектом исследования является микросервисная архитектура высоконагруженных систем электронной коммерции.

Предметом исследования являются методы и протоколы передачи данных (REST, gRPC, GraphQL) и их влияние на производительность системы при различных профилях сетевой нагрузки.

Цель работы заключается в повышении эффективности обработки запросов в E-commerce приложениях путем разработки гибридной архитектуры, использующей паттерн API Gateway с GraphQL для агрегации данных и протокол gRPC для межсервисного взаимодействия.

Для достижения поставленной цели необходимо решить следующие задачи:
\begin{enumerate}
	\item Провести анализ существующих архитектурных паттернов (BFF, API Gateway) и протоколов передачи данных, применяемых в высоконагруженных системах.
	\item Разработать архитектуру программного комплекса, реализующую паттерн API Gateway с поддержкой протоколов GraphQL и gRPC.
	\item Провести нагрузочное тестирование и сравнительный анализ производительности разработанных решений в условиях идеальной сети и сетевых ограничений.
	\item Оценить эффективность сжатия данных и влияние выбора протокола на время отклика системы и потребление ресурсов.
\end{enumerate}


%% Вспомогательные команды - Additional commands
%\newpage % принудительное начало с новой страницы, использовать только в конце раздела
%\clearpage % осуществляется пакетом <<placeins>> в пределах секций
%\newpage\leavevmode\thispagestyle{empty}\newpage % 100 % начало новой строки